\usepackage[ngerman]{babel} % language support
\usepackage[utf8]{inputenc}
\usepackage{graphicx} % displaying images
\usepackage{scrlayer-scrpage}
\usepackage[german=quotes]{csquotes} % quote
\usepackage{ulem}
\usepackage{multirow} % use of multiple rows and columns
\usepackage{tablefootnote} % footnote in tables
% \usepackage{svg} % use of svg
\usepackage[table,xcdraw]{xcolor}  
\usepackage{listings} % Use of code
\usepackage{xcolor} % Color of codings
\usepackage{array} % Alignment of table columns
\usepackage{longtable}  %span table over multiple pages
\usepackage[printonlyused,withpage]{acronym}

\usepackage{tikz} %for checkmark
\def\checkmark{\tikz\fill[scale=0.4](0,.35) -- (.25,0) -- (1,.7) -- (.25,.15) -- cycle;}

\useunder{\uline}{\ul}{}

% Design of Codeblock
\definecolor{mygreen}{rgb}{0,0.6,0}
\definecolor{mygray}{rgb}{0.5,0.5,0.5}
\definecolor{mymauve}{rgb}{0.58,0,0.82}

\lstset{
    backgroundcolor=\color{white},   % choose the background color; you must add \usepackage{color} or \usepackage{xcolor}; should come as last argument
    basicstyle=\footnotesize,        % the size of the fonts that are used for the code
    breakatwhitespace=false,         % sets if automatic breaks should only happen at whitespace
    breaklines=true,                 % sets automatic line breaking
    captionpos=b,                    % sets the caption-position to bottom
    commentstyle=\color{mygreen},    % comment style
    extendedchars=true,              % lets you use non-ASCII characters; for 8-bits encodings only, does not work with UTF-8
    firstnumber=1,                   % start line enumeration with line 1000
    frame=single,	                   % adds a frame around the code
    keepspaces=true,                 % keeps spaces in text, useful for keeping indentation of code (possibly needs columns=flexible)
    keywordstyle=\color{blue},       % keyword style
    language=SQL,                 % the language of the code
    morekeywords={foaf, rel, base, rdf},            % if you want to add more keywords to the set
    numbers=left,                    % where to put the line-numbers; possible values are (none, left, right)
    numbersep=5pt,                   % how far the line-numbers are from the code
    numberstyle=\color{mygray}, % the style that is used for the line-numbers
    rulecolor=\color{black},         % if not set, the frame-color may be changed on line-breaks within not-black text (e.g. comments (green here))
    showspaces=false,                % show spaces everywhere adding particular underscores; it overrides 'showstringspaces'
    showstringspaces=false,          % underline spaces within strings only
    showtabs=false,                  % show tabs within strings adding particular underscores
    stepnumber=1,                    % the step between two line-numbers. If it's 1, each line will be numbered
    stringstyle=\color{mymauve},     % string literal style
    tabsize=2,	                   % sets default tabsize to 2 spaces
    title=\lstname                   % show the filename of files included with \lstinputlisting; also try caption instead of title
}


% definition of own Table column style which is centered vertically and horizontally
\newcolumntype{M}[1]{>{\centering\arraybackslash}m{#1}}

% forcing placement of figures
\usepackage{float}

% Changing emphasize style and force wrapping in bibliography
\usepackage{ulem}

% Reference-packages, have to be at the end of 'imports'
\usepackage{hyperref}
\usepackage{cleveref}


\graphicspath{images}
\linespread{1.25}

% Remove big spacing of chapter beginning
\renewcommand*{\chapterheadstartvskip}{\vspace*{0cm}}

\counterwithout{footnote}{chapter}

% Creation of Bibliography
\newcommand{\settingBibFootnoteCite}{
    \setlength{\bibparsep}{\parskip}		  % Add some space between biblatex entries in the bibliography
    \addbibresource{bibliography.bib}	    % Add file bibliography.bib as biblatex resource
    \setcounter{biburlnumpenalty}{9000}
    \setcounter{biburlucpenalty}{9000}
    \setcounter{biburllcpenalty}{9000}
    \DefineBibliographyStrings{ngerman}{andothers = {{et\,al\adddot}},}
}


\newcommand{\initializeBibliography}{
    \ihead{}
    \normalem
    \printbibliography[title=Literaturverzeichnis]
}